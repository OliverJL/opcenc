\documentclass[11pt]{scrartcl}
%\documentclass{article}
\usepackage[utf8]{inputenc}
\usepackage[english]{babel}
\usepackage{amsmath}
\usepackage{titlesec}
\usepackage[utf8]{inputenc}
\usepackage[T1]{fontenc}
%\usepackage{lmodern}
\usepackage{multicol}
%\setlength{\columnsep}{1cm}

% Title Page
\title{Analysis of Opcode Encryption}
\date{5. Januar 2004}
\author{Oliver Jessl}

\begin{document}

\maketitle
\tableofcontents

\begin{abstract}
Großbritannien ist das Land der 1001 Tabus, das größte Tabu ist Geld. Ein Brite würde sich eher eine Schrotladung ins Knie jagen, als freiwillig über seine Finanzlage zu reden. Insofern kommt es einem politischen Vulkanausbruch gleich, was David Cameron am Wochenende tat: Er veröffentlichte eine Zusammenfassung seiner Steuererklärung aus den vergangenen sechs Jahren.
\end{abstract}

\section{INTRODUCTION}

 - attacker intercepts cryptopgram. He is able to determine posteriori probabilities of the various possible messages and keys which produced to the cryptogramm[1, S. 3]
 - find degree of encryption at which avalance effect is reached.
 - Considering a secrecy system to be a set of trans-
 formations of one space into another with definite probabilities
 associated with each transformation, there are two natural combining
 operations which produce a third system from two given systems.
 The first combining operation is called the product operation and
 corresponds to enciphering the message with the first system R and
 enciphering the resulting cryptogram with system S; the keys for R
 and S being chosen independently. This total operation is a secrecy
 system whose transformations consist of all the products (in the
 usual sense of products of transformations) of transformations in S
 with transformations in R. The probabilities are the products of the
 probabilities for the two transformations. [1, S.3]
 
  

\subsection{Introduction AES}

\subsection{Chiphermodes}

\subsubsection{ECB}
 weak \cite{todo}
\subsubsection{ECB}

\subsection{Assumptions}
All algorithms are implemented in the same quality. A difference in resource consumption is attributable to efficiency of the algorithm, but not the result of different implementation qualities.

\section{MATERIALS AND METHODS}

- Performance Analysis



\section{References TOOLS}

OpenSSL 1.0.2
The OpenSSL Project
https://github.com/openssl/openssl/tree/OpenSSL_1_0_2-stable





\begin{thebibliography}{laengste Labelbreite}

  \bibitem[amtoc]{amtoc} Shannon, Claude E. "A mathematical theory of cryptography." Memorandum MM 45 (1945): 110-02.
  
  \bibitem[cacp]{cacp} Feistel, Horst. "Cryptography and computer privacy." Scientific american 228 (1973): 15-23.
   
  \bibitem[paamea]{paamea}Mohan, H. S., and A. Raji Reddy. "Performance analysis of AES and MARS encryption algorithms." IJCSI International Journal of Computer Science Issues 8.4 (2011): 1694-0814.
  
  \bibitem[aeaes]{aeaes}  Bhoge, Jayant P., and Prashant N. Chatur. "Avalanche Effect of AES Algorithm." IJCSIT International Journal of Computer Science and Information Technologies 5.3 (2014): 3101-3103.

  \bibitem[opsll]{opsll} OpenSSL 1.0.2, “OpenSSL:The Open Source toolkit for SSL/TSL,”, The OpenSSL Project, http://www.openssl.org/

  \bibitem[todo]{todo}TODO
% __builtin_popcountll Hamming 
  \bibitem[]{gccpcl}https://gcc.gnu.org/onlinedocs/gcc/Other-Builtins.html (__builtin_popcountll)
% Practical Embedded Security: Building Secure Resource-Constrained Systems  
\end{thebibliography}

\end{document}