\documentclass[11pt]{scrartcl}
%\documentclass{article}
\usepackage[utf8]{inputenc}
\usepackage[english]{babel}
\usepackage{amsmath}
\usepackage{titlesec}
\usepackage[utf8]{inputenc}
\usepackage[T1]{fontenc}
%\usepackage{lmodern}



\usepackage{multicol}
%\setlength{\columnsep}{1cm}



% Title Page
\title{Analysis of Opcode Encryption}
\date{5. Januar 2004}
\author{Oliver Jessl}


\begin{document}

\maketitle
\tableofcontents





\begin{abstract}
Großbritannien ist das Land der 1001 Tabus, das größte Tabu ist Geld. Ein Brite würde sich eher eine Schrotladung ins Knie jagen, als freiwillig über seine Finanzlage zu reden. Insofern kommt es einem politischen Vulkanausbruch gleich, was David Cameron am Wochenende tat: Er veröffentlichte eine Zusammenfassung seiner Steuererklärung aus den vergangenen sechs Jahren.
\end{abstract}

%\begin{multicols}{2}

\section{INTRODUCTION dsdf sdf sdfsljklkk sdf}

Hier kommt die Einleitung. Ihre Überschrift kommt
automatisch in das Inhaltsverzeichnis.

\subsection{Formeln}

\LaTeX{} ist auch ohne Formeln \cite{paamea} sehr nützlich und
einfach zu verwenden. Grafiken, Tabellen,
Querverweise aller Art, Literatur- und
Stichwortverzeichnis sind kein Problem.

Formeln sind etwas schwieriger, dennoch hier ein
einfaches Beispiel.  Zwei von Einsteins
berühmtesten Formeln lauten:
\begin{align}
E &= mc^2                                  \\
m &= \frac{m_0}{\sqrt{1-\frac{v^2}{c^2}}}
\end{align}
Aber wer keine Formeln schreibt, braucht sich
damit auch nicht zu beschäftigen.

%\end{multicols}

\begin{thebibliography}{laengste Labelbreite}
   \bibitem[1]{paamea}"Mohan, H. S., and A. Raji Reddy. "Performance analysis of AES and MARS encryption algorithms." IJCSI International Journal of Computer Science Issues 8.4 (2011): 1694-0814."
\end{thebibliography}

\end{document}